% \documentclass{standalone}
\documentclass{article}

\usepackage{tikz}
\usetikzlibrary{calc}

\usepackage{ifthen}


\newcommand{\checkNumberModulo}[2]{%
  \ifthenelse{\modulo{#1}{#2}=0}%
  {\newline\vspace{4cm}}% If number is zero
    {\hspace{1cm}}% If number is not zero
  }

% Modulo command
\makeatletter
\newcommand{\modulo}[2]{%
  \number\numexpr#1-#2*(\number\numexpr(#1/#2)\relax)\relax
}
\makeatother

\begin{document}

\tikzset{
    every node/.style={
        circle,
        draw,
        solid,
        fill=black,
        inner sep=0pt,
        minimum width=3pt
    }
}


\def\mod{3};

\foreach \i in {1,2,3,4,5,6,7,8,9,10,11,12,13,14}{
\begin{tikzpicture}[scale=0.7,baseline]
  \def\prev{0}
\def\numbers{1}
\def\angle{0}
\coordinate (\prev) at (1,0);

  \foreach \curr in \numbers{
  \coordinate (\curr) at ($ (\prev) + (\angle:1) $);
  \draw (\prev) node {} -- (\curr) node {};
  \xdef\prev{\curr} % Update the previous element to the current element
}

  \foreach \j in {1,2,3,4}{
    \input{tikz/path_\i_\j.tikz}
    \foreach \curr in \numbers{
  \coordinate (\curr) at ($ (\prev) + (\angle:1) $);
  \draw (\prev) node {} -- (\curr) node {};
  \xdef\prev{\curr} % Update the previous element to the current element
}
}
\end{tikzpicture}
\checkNumberModulo{\i}{\mod}}
% \checkNumber{\modulo{\i}{4}}}

% \end{minipage}
  

\end{document}